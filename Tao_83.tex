\documentclass{article}
\usepackage{verbatim}
\usepackage[utf8x]{inputenc}
\title{PERFECT NUMBERS\footnotetext{Originally published in \emph{Trigon} (School Mathematics Journal of the Mathematical Association of South Australia) \textbf{21} (3), Nov. 1983, p. 7.}}
\date{}
\author{}

\pagestyle{empty}

\begin{document}


\maketitle
\thispagestyle{empty}

\noindent
A \textbf{perfect number} is one such that all its factors, including one but excluding itself, add up to itself. For example, $6$ is a perfect number since $6$ has factors $1, 2, 3$ and $6$ and $1 + 2 + 3 = 6$. In fact, $6$ is the smallest perfect number. 

\textbf{Euclid} proved in his \textbf{Elements} that a number of the form $2^{p-1}(2^p - 1)$ is a perfect number if $2^p - 1$ is a prime number.

I used this fact to write a programme in Basic to find perfect numbers but first we need a programme on prime numbers for checking if $2^p - 1$ is prime.

\begin{verbatim}

      ready.
      10 rem prime numbers 
      11 rem to calculate prime numbers up to a 
      20 input a 
      22 if a=2 then print"2": goto 100 
      25 print"2 3"; 
      30 for i=2 to a 
      40 if i=a then 100 
      50 for d=2 to int(sqr(i)+2) 
      60 if i/d=int(i/d) then 90 
      70 next d
      80 printi;
      90 next i
      100 end

\end{verbatim}

\noindent So now let us see how we can use lines 40-60 to find perfect numbers. 

\begin{verbatim}

      10 rem perfect numbers
      15 rem to calculate perfect numbers
      20 input n
      30 if n<6 then print "none":goto 200
      35 if n=6 then print "6 only":goto 200
      40 print"6"; 
      45 for i=3 to 26 
      46 rem limit n to 2↑25*(2↑26-1)
      47 let y=2↑i-1 
      50 rem next loop is to check if 2↑i-1 is prime
      52 for l=2 to int(sqr(y))
      53 if y/l=int(y/l) then 70
      54 if y*2↑(i-1)>n then 200
      55 next l
      57 print",";y*2↑(i-1);
      70 next i 
      200 print 
      201 print"(this program was written on 26/8/83)" 
      300 end 

\end{verbatim}

Unfortunately, line 45 limits us to $2^{25} (2^{26} - 1)$, but then the computer has a limited range of numbers: it will never get to $2^{25} (2^{26} - 1)$ anyway. I have computed perfect numbers up to $10^{13}$. 

\[
6, 28,496, 8128, 33\,550\,336, 8.58986906\mathrm{e} + 09, 1.37438691\mathrm{e} + 11 
\]

The last two, of course, are only approximations to the actual perfect numbers and are unacceptable in this form. 

\noindent $8.58986906\mathrm{e} + 09 = 8\,589\,869\,060$ when the last two figures are in doubt. In fact it is $8\,589\,869\,056$. 

\medskip
\begin{flushright}
\textbf{Terence Tao}
\end{flushright}

\end{document}